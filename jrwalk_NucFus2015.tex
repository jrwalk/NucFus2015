\documentclass[12pt]{iopart}

\usepackage{iopams}
\usepackage{graphicx}
\usepackage{subfig}
\usepackage{color}
\usepackage{multirow}

\newcommand{\eg}{\emph{e.g., }}
\newcommand{\ie}{\emph{i.e., }}
\renewcommand{\etal}{\emph{et al. }}

% remove these for final publication 
\usepackage{color} 
\newcommand{\gnote}[1]{\marginpar{\textcolor{red}{\scriptsize{#1}}}}
\renewcommand{\note}[1]{\textcolor{red}{#1}}

\graphicspath{{./graphics/}{./pdfgraphics/}}  % remove pdfgraphics path for final publication!

\begin{document}

\title[Global Performance and Confinement in I-mode]{Impact of the Pedestal on Global Performance and Confinement Scalings in I-Mode}
\author{J. R. Walk$^{1}$, J. W. Hughes$^{1}$, A. E. Hubbard$^{1}$, F. Ryter$^{2}$, D. G. Whyte$^{1}$, A. E. White$^{1}$}
\address{$^1$ MIT Plasma Science and Fusion Center, 77 Massachusetts Avenue, Cambridge, MA 02139}
\address{$^2$ Max-Planck-Institut f\"ur Plasmaphysik}
\eads{jrwalk@psfc.mit.edu}

\begin{abstract}
I-mode is a novel alternate high-confinement tokamak regime, notable for the formation of a strong temperature pedestal with associated H-mode-like increase in energy confinement, without the accompanying density pedestal or suppression of particle transport.
I-mode exhibits a number of attractive features for a tokamak reactor regime, namely (1) an inherent lack of large, deleterious Edge-Localized Modes (ELMs), (2) minimal impurity accumulation and radiative loss compared to conventional H-modes, and (3) an apparent lack of strong degradation of energy confinement with input heating power.  
Previous analyses of I-mode experiments at Alcator C-Mod have elucidated the pedestal structure in I-mode, particularly in its strong positive response to fueling and input heating power.
Global performance and confinement responds accordingly to these inputs, with both absolute (\eg plasma stored energy) and normalized (\eg $\beta_N$) metrics responding strongly to fueling and heating power.
Due to core temperature profile stiffness \note{further effects?}, the very high pedestal temperature in I-mode results in comparable core and global-averaged pressures to H-mode despite the relaxed density profile, although moderate levels of density peaking are still observed.
The minimal degradation of energy confinement time with heating power in I-mode is also observed empirically, in contrast to the strong ($\tau_E \sim P^{-0.7}$) degradation found in H-mode.
Following the practices of the multi-machine ITER89 and ITER98 confinement scalings for L-mode and ELMy H-mode respectively, an initial assay at a confinement scaling for I-mode is also presented.
The single-machine scaling captures the physics observed on Alcator C-Mod \note{ASDEX data influence?}
\note{something something informs extrapolation to ITER}
\end{abstract}

\pacs{52.55.Fa,52.55.Tn,52.25.Fi,52.40.Hf,52.35.Py}

\maketitle

\section{Introduction}\label{sec:intro}

Viable \& economical power generation via tokamak magnetic-confinement fusion is characterized by two seemingly-contradictory requirements.
First, a high level of energy confinement is necessary for the desired level of self-heating by fusion products (\eg ``alpha heating'' for deuterium-tritium fusion) for net energy production.  
At the same time, particle confinement must be sufficiently low to avoid the deleterious effects of accumulated impurities (both the helium ``fusion ash'' and high-$Z$ materials introduced from eroded plasma-facing surfaces) due to fuel dilution and radiative losses.
The energy confinement requirement has been achieved in a class of operating regimes, collectively termed ``H-modes'' \cite{Wagner1982}, characterized by the formation of an edge transport barrier, termed the \emph{pedestal}, characterized by steep gradients in density, temperature, and pressure, with H-modes capable of stationary operation requiring a relaxation mechanism on the density pedestal to avoid impurity accumulation.

While the height of the pedestal is strongly correlated with global fusion performance \cite{Kinsey2011}, the strong gradients inherent in the pedestal have been shown to drive edge MHD instabilities \cite{Huysmans2005,Maget2013,Snyder2002} resulting in an Edge-Localized Mode (ELM), an explosive perturbation to the pedestal driving a rapid burst of energy and particle transport into the plasma exhaust \cite{Zohm1996}.  
On existing devices, these bursts of transport are sufficient to vent accumulated impurities from the plasma, allowing stationary operation with acceptable radiative losses \cite{Keilhacker1984}.
As the ELMy H-mode is robust and relatively straightforward to achieve, the regime is considered the baseline for high-confinement operation on ITER \cite{ITER1999,Shimada2007}.  
However, on ITER- or reactor-scale devices ELMs drive transient heat loads to the divertor, leading to unacceptable levels of erosion and heat damage to plasma-facing components \cite{Loarte2003,Federici2003}.
As such, the avoidance, suppression, or mitigation of large ELMs, either via externally-applied engineering solutions (pellet pacing \cite{Baylor2013,Lang2014} or resonant magnetic perturbations \cite{Evans2004,Evans2006}), or via alternate high-confinement regimes which regulate the pedestal below the ELM stability limit (\eg the Enhance $D_\alpha$ (EDA) H-mode \cite{Greenwald1999,Hubbard2001} or QH-mode \cite{Burrell2002,Suttrop2005}).

The I-mode \cite{Whyte2010,Hubbard2011,Walk2014,Walk2014b}, pioneered on the Alcator C-Mod tokamak \cite{Hutchinson1994}, is a promising new regime for high-performance operation.  
I-mode is unique among high-performance regimes in that it evidently decouples energy and particle transport -- the regime develops a strong temperature pedestal and associated H-mode-like energy confinement improvement, while retaining an L-mode-like density profile lacking a pedestal or suppression of particle transport as in conventional H-modes (see Figure~\ref{fig:imode_trace}).
I-mode operation offers several attractive features for a reactor regime: (1) relaxed density profile providing desirable L-mode-like levels of impurity accumulation \cite{Howard2011}, (2) an inherent lack of large ELMs, avoiding the need for externally-applied ELM control, and (3) minimal degradation of energy confinement with heating power \cite{Whyte2010,Dominguez2012}, in contrast to the degradation found in ELMy H-mode (roughly $\tau_E \sim P^{-0.7}$ \cite{ITER1999,Christiansen1992}).



\begin{figure}[ht]
 \centering
 \includegraphics[width=\textwidth]{pdfgraphics/trace_imode.pdf}
 \caption{(left) characteristic time traces for an I-mode.  After the L-I transition, the core and edge temperature rise over several sawtooth cycles (visible in the oscillations in $T_e(0)$ and $T_{e,ped}$) before reaching a steady level; global pressure and confinement rise accordingly.  However, the density remains unchanged from the L-mode level.  No ELMs are exhibited on the $D_\alpha$ trace.  (right) pedestal profiles for L-, I-, and H-modes.  The I-mode (green) retains a density profile similar to L-mode (black), unlike the ELMy (red) and EDA (blue) H-modes, which form a strong density pedestal.  However, the I-mode forms a higher temperature pedestal than either H-mode, resulting in comparable pedestal pressures to H-mode while retaining L-mode particle transport.}
 \label{fig:imode_trace}
\end{figure}

\section*{References}
\bibliographystyle{unsrt}
\bibliography{jrwalk_references}

\end{document}