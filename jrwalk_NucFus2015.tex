\documentclass[12pt]{iopart}

\usepackage{iopams}
\usepackage{graphicx}
\usepackage{subfig}
\usepackage{color}
\usepackage{multirow}

\newcommand{\eg}{\emph{e.g., }}
\newcommand{\ie}{\emph{i.e., }}
\renewcommand{\etal}{\emph{et al. }}

% remove these for final publication 
\usepackage{color} 
\newcommand{\gnote}[1]{\marginpar{\textcolor{red}{\scriptsize{#1}}}}

\graphicspath{{./graphics/}{./pdfgraphics/}}  % remove pdfgraphics path for final publication!

\begin{document}

\title[Global Performance and Confinement in I-mode]{Impact of the Pedestal on Global Performance and Confinement Scalings in I-Mode}
\author{J. R. Walk$^{1}$, J. W. Hughes$^{1}$, A. E. Hubbard$^{1}$, F. Ryter$^{2}$, D. G. Whyte$^{1}$, A. E. White$^{1}$}
\address{$^1$ MIT Plasma Science and Fusion Center, 77 Massachusetts Avenue, Cambridge, MA 02139}
\address{$^2$ Max-Planck-Institut f\"ur Plasmaphysik}
\eads{jrwalk@psfc.mit.edu}

\begin{abstract}
I-mode is a novel alternate high-confinement tokamak regime, notable for the formation of a strong temperature pedestal with associated H-mode-like increase in energy confinement, without the accompanying density pedestal or suppression of particle transport.
I-mode exhibits a number of attractive features for a tokamak reactor regime, namely (1) an inherent lack of large, deleterious Edge-Localized Modes (ELMs), (2) minimal impurity accumulation and radiative loss compared to conventional H-modes, and (3) an apparent lack of strong degradation of energy confinement with input heating power.  
Previous analyses of I-mode experiments at Alcator C-Mod have elucidated the pedestal structure in I-mode, particularly in its strong positive response to fueling and input heating power.
\end{abstract}

\pacs{52.55.Fa,52.55.Tn,52.25.Fi,52.40.Hf,52.35.Py}

\maketitle

\section{Introduction}\label{sec:intro}

bar \etal \cite{Walk2014}

\section*{References}
\bibliographystyle{unsrt}
\bibliography{jrwalk_references}

\end{document}